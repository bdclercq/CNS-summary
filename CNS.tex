\documentclass[12pt]{article}
\usepackage{graphicx} % This lets you include figures
\usepackage{hyperref} % This lets you make links to web locations
\graphicspath{ {./images/} }

\usepackage[rightcaption]{sidecap}
\usepackage{caption}
\usepackage{subcaption}
\usepackage{hyperref}

\usepackage{float}

\usepackage{imakeidx}

\makeindex


\title{Computer and network security}
\author{Beau De Clercq}
\date{2020-2021}

\begin{document}
	
 \maketitle{}
 
 \tableofcontents
 
 \clearpage
 \newpage
 
 \section{Introduction}
 
 \section{Symmetric ciphers}
 
 \section{Message authentication}
 \subsection{Hash functions}
 A hash function H is a function that takes input data blocks of length M and returns a hash value of fixed size R.\\
 A cryptographic hash function that also satisfies following conditions:
 \begin{itemize}
 	\item One way property: it should be infeasible to find a data object that maps to a predefined hash value.
 	\item Collision free property: it should be infeasible to find 2 data objects that map to the same hash value.
 	\item Use padding to pad up input to fixed length and add the length l of the block in bits. 
 \end{itemize}
By satisfying the first two properties, hash functions can  be used to determine if data has been altered.\\
Hash functions can be used in an number of applications:
\begin{itemize}
	\item Message authentication: to ensure a message hasn't been altered.
	\item Digital signatures: ensure the authenticity of messages and identity of the sender.
	\item One-way password file: store hash value of password in plain text file.
	\item Intrusion/virus detection: store H(f) for each file to determine if files have been modified.
	\item Pseudorandom function: use H to generate pseudorandom private key.
\end{itemize}

 
 \subsection{Secure Hash Algorithm (SHA)}
 \subsection{Length extension attack and SHA3}
 \subsection{Message authentication}
 \subsection{Message authentication codes}
 
 \section{Asymmetric encryption}
 
 \section{Key distribution}
  
\end{document}